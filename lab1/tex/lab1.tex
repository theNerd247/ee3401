\documentclass[main.tex]{subfile}

\begin{document}

\section{Wave Form Generator}
\label{sec:wave_form_generator}

Wave form generators are electrical devices that generate an electrical output
signal of a defined pattern - sinusoidal, square, etc. A wave form generator has
multiple properties for configuring the output as follows:

\begin{itemize}
	\item \textbf{Output type}. This allows the user to switch between sine, square, pulse,
		and other output wave forms.
	\item \textbf{Peak-to-peak voltage}. The waveform generator has a high-impedance circuit
		that doubles the amplitude of the waveform output. Therefore the user must set
		the output voltage in the waveform generator as the peak-to-peak voltage to
		compensate for this modified output.
	\item \textbf{Output frequency}. The output frequency can be set in terms of
		hertz.
\end{itemize}

\section{Sample Waveform Generator Settings}
\label{sec:sample_waveform_generator_settings}

Consider a wave shown in \eqref{sampleWave} that is connected to a high-impedance
load of $1\texttt{M}\Omega$. The ideal and real settings of the waveform-generator
are shown in \tabref{genSettings}.

\begin{equation}
	\label{eq:sampleWave}
	y = 5\sin({120{\pi}t}) 
\end{equation}

\begin{table}[h]
	\begin{center}
		\caption{Sample Waveform Generator Settings}
		\label{tab:genSettings}
		\begin{tabular}{llcc}
			\\ \toprule
			Output type & Amplitude ($\texttt{V}$) & Frequency ($\texttt{Hz}$) & Phase
			Shift
			\\ \midrule
			Ideal & $5$ (not peak-to-peak) & $60$ & 0
			\\ Real  & $5$ (peak-to-peak) & $60$ & 0
			\\ \bottomrule
		\end{tabular}
	\end{center}
\end{table}

% section sample_waveform_generator_settings (end)

% section wave_form_generator (end)

\section{Oscilloscopes}
\label{sec:oscilloscope}

Oscilloscopes are tools used to visualize an electric signal. Typically they
allow multiple signals to be visualized at the same time, and modifications to
the display such as the time scale, the wave amplitude scale, the sampling
frequency, and the signal triggering methods. The oscilloscopes that we will be
using has other features such as computing analytical properties of an input
waveform, and saving the display as an image. Below are details of each of the
features of an oscilloscope:

\begin{itemize}
	\item \textbf{Time scale}. This modifies the x-axis scale of the waveform
		plot. Typically this is adjusted to show only $1-2$ cycles of the input waveform.
	\item \textbf{Signal triggering}. This sets the mechanism for how the
		oscilloscope detects the signal and rendering timing. Typically this is
		adjusted such that the input signal will be "stable" - that is not moving
		sideways on the screen.
	\item \textbf{Image Saving}. The digital oscilloscope also has the ability
		to save a display to a file on a USB drive.

		\begin{figure}[!t]
			\centering
			\includegraphics[width=.75\linewidth]{lab1.jpg}
			\caption{Sample Saved Oscilloscope Image}
			\label{fig:sampleImage}
		\end{figure}

	\item \textbf{Noise filtering}. The oscilloscope includes a Low-pass filter
		and signal averaging functions that are used to filter out noise in the
		incoming signal, this is used to clean up a bit of the signal that is
		displayed.
	\item \textbf{Measuring Tools}. The digital oscilloscope used in the lab has
		the ability to measure analytical properties of a waveform. Such as wave
		amplitude (both regular and peak-to-peak), period, frequency, etc. Measured
		properties of the generated wave in \eqref{sampleWave} is shown in
		\tabref{measuredProp}

		\begin{table}[H]
			\begin{center}
				\caption{Measured Properties of Sample Wave}
				\label{tab:measuredProp}
				\begin{tabular}{ll}
					\\ \toprule
					Property & Value 
					\\ \midrule
					Frequency & $59.9\texttt{Hz}$
					\\Peak-to-Peak & $9.93\texttt{V}$
					\\Minimum Voltage & $4.94\texttt{V}$
					\\Maximum Voltage & $9.93\texttt{V}$
					\\ \bottomrule
				\end{tabular}
			\end{center}
		\end{table}

\end{itemize}

% section oscilloscope (end)

\section{Conclusion} 
\label{sec:conclusion}

In this lab we played with the measuring features provided by the oscilloscope
with input directly from a waveform generator. The user interface is intuitive
enough for working in the lab this semester.
% section conclusion (end)

\end{document}
