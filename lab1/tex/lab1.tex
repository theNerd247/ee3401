\documentclass[main.tex]{subfile}

\begin{document}

\section{Wave Form Generator} 
\label{sec:wave_form_generator}

Wave form generators are electrical devices that generate an electrical output
signal of a defined pattern - sinusoidal, square, etc. A wave form generator has
multiple properties for configuring the output as follows: 

\begin{itemize}
\item \textbf{Output type}. This allows the user to switch between sine, square, pulse,
	and other output wave forms. 
\item \textbf{Peak-to-peak voltage}. The waveform generator has a high-impedance circuit
	that doubles the amplitude of the waveform output. Therefore the user must set
	the output voltage in the waveform generator as the peak-to-peak voltage to
	compensate for this modified output.
\item \textbf{Output frequency}. The output frequency can be set in terms of
	hertz.
\end{itemize}

\section{Sample Waveform Generator Settings} 
\label{sec:sample_waveform_generator_settings}

For a wave of the form $5\sin({120{\pi}t})$ that is connected to a high-impedance
load of $1\texttt{M}\Omega$ the ideal and real settings of the waveform-generator
are shown in \tabref{genSettings}.

\begin{table}[H]
  \begin{center}
    \caption{Sample Waveform Generator Settings}
    \label{tab:genSettings}
    \begin{tabular}{llcc}
      \\ \toprule
			Output type & Amplitude ($\texttt{V}$) & Frequency ($\texttt{Hz}$) & Phase
			Shift
      \\ \midrule
			Ideal & $5$ (not peak-to-peak) & $60$ & 0
			\\ Real  & $5$ (peak-to-peak) & $60$ & 0
      \\ \bottomrule
    \end{tabular}
  \end{center}
\end{table}

% section sample_waveform_generator_settings (end)

% section wave_form_generator (end)

\section{Oscilloscopes} 
\label{sec:oscilloscope}

Oscilloscopes are tools used to visualize an electric signal. Typically they
allow multiple signals to be visualized at the same time, and modifications to
the display such as the time scale, the wave amplitude scale, the sampling
frequency, and the signal triggering methods. The oscilloscopes that we will be
using have other features such as computing analytical properties of an input
waveform, and saving the display as an image. Below are details of each of the
features of an oscilloscope: 

\begin{itemize}
		\item \textbf{Channel selection}. There are four input channels to select
			from, each with independent property controls (amplitude, time scale,
			etc).
		\item \textbf{Time scale}. This modifies the x-axis scale of the waveform
			plot. Typically this is adjusted to show only $1-2$ cycles of the input waveform.
		\item \textbf{Signal triggering}. This enables 
\end{itemize}


\begin{figure}[H]
	\begin{center}
		\includegraphics[width=\linewidth]{lab1.jpg}
	\end{center}
	\caption{Sample Saved Oscilloscope Image}
	\label{fig:sampleImage}
\end{figure}

% section oscilloscope (end)

\end{document}
